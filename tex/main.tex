%                                                                 aa.dem
% AA vers. 9.1, LaTeX class for Astronomy & Astrophysics
% demonstration file
%                                                       (c) EDP Sciences
%-----------------------------------------------------------------------
%
%\documentclass[referee]{aa} % for a referee version
%\documentclass[onecolumn]{aa} % for a paper on 1 column  
%\documentclass[longauth]{aa} % for the long lists of affiliations 
%\documentclass[letter]{aa} % for the letters 
%\documentclass[bibyear]{aa} % if the references are not structured 
%                              according to the author-year natbib style

%
\documentclass{aa}  

%
\usepackage{graphicx}
%%%%%%%%%%%%%%%%%%%%%%%%%%%%%%%%%%%%%%%%
\usepackage{txfonts}
\usepackage[svgnames]{xcolor}
%%%%%%%%%%%%%%%%%%%%%%%%%%%%%%%%%%%%%%%%
% \usepackage[options]{hyperref}
% To add links in your PDF file, use the package "hyperref"
% with options according to your LaTeX or PDFLaTeX drivers.
%
\usepackage{hyperref}         % automagic cross-referencing
\usepackage{cleveref}
\newcommand\numberthis{\addtocounter{equation}{1}\tag{\theequation}}
% defines the color of hyperref objects
% Blending two colors:  blue!80!black  =  80% blue and 20% black
\hypersetup{ % this is just my personal choice, feel free to change things
    colorlinks,
    linkcolor={red!50!black},
    citecolor={blue!20!purple!80!black},
    urlcolor={blue!80!black},
    breaklinks=true}
\urlstyle{same}
\begin{document} 


   \title{TITLE}

   \subtitle{AST5220}

   \author{Håvard Skåli
        %   \inst{1}
          }

   \institute{Institute of Theoretical Astrophysics (ITA), University of Oslo}

   \date{\today}

% \abstract{}{}{}{}{} 
% 5 {} token are mandatory
 
  \abstract
%   % context heading (optional)
%   % {} leave it empty if necessary  
%    {To investigate the physical nature of the `nuc\-leated instability' of
%    proto giant planets, the stability of layers
%    in static, radiative gas spheres is analysed on the basis of Baker's
%    standard one-zone model.}
%   % aims heading (mandatory)
%    {It is shown that stability
%    depends only upon the equations of state, the opacities and the local
%    thermodynamic state in the layer. Stability and instability can
%    therefore be expressed in the form of stability equations of state
%    which are universal for a given composition.}
%   % methods heading (mandatory)
%    {The stability equations of state are
%    calculated for solar composition and are displayed in the domain
%    $-14 \leq \lg \rho / \mathrm{[g\, cm^{-3}]} \leq 0 $,
%    $ 8.8 \leq \lg e / \mathrm{[erg\, g^{-1}]} \leq 17.7$. These displays
%    may be
%    used to determine the one-zone stability of layers in stellar
%    or planetary structure models by directly reading off the value of
%    the stability equations for the thermodynamic state of these layers,
%    specified
%    by state quantities as density $\rho$, temperature $T$ or
%    specific internal energy $e$.
%    Regions of instability in the $(\rho,e)$-plane are described
%    and related to the underlying microphysical processes.}
%   % results heading (mandatory)
%    {Vibrational instability is found to be a common phenomenon
%    at temperatures lower than the second He ionisation
%    zone. The $\kappa$-mechanism is widespread under `cool'
%    conditions.}
%   % conclusions heading (optional), leave it empty if necessary 
%    {}

%    \keywords{giant planet formation --
%                 $\kappa$-mechanism --
%                 stability of gas spheres
%                }
{AIMS}
{METHODS}
{RESULTS}
{}\keywords{KEYWORDS}


  % This paper presents a cosmological N-body simulation to investigate the formation and evolution of dark matter halos across cosmic time, using a 20 Mpc simulation box with \(64^3\) particles. The simulation covers redshifts from \( z \approx 12.9 \) to \( z = 0 \), allowing for analysis of the halo mass function (HMF) and dark matter density profiles at various stages. The HMF results are fitted with the Press-Schechter and Sheth-Tormen models, with the latter providing a closer match to high-mass halos due to its consideration of ellipsoidal collapse. At \( z = 0 \), the Sheth-Tormen model predicts a number density for halos of \( \sim 10^{-2} \, h^2 \, \text{Mpc}^{-3} \) for masses around \( 10^{12} \, M_\odot \), closely matching the simulation data. Additionally, density profiles were well-represented by the Navarro-Frenk-White profile, showing “cuspy” central regions in line with theoretical predictions. Comparison with a larger simulation (40 Mpc, \(128^3\) particles) confirmed that higher volumes reduce sampling variance, especially for massive halos. This project's findings support the hierarchical model of structure formation and highlight the importance of resolution and volume in dark matter simulations.

   \maketitle
%
%-------------------------------------------------------------------

\section{Milestone I}

%    In the \emph{nucleated instability\/} (also called core
%    instability) hypothesis of giant planet
%    formation, a critical mass for static core  envelope
%    protoplanets has been found. \citet{mizuno} determined
%    the critical mass of the core to be about $12 \,M_\oplus$
%    ($M_\oplus=5.975 \times 10^{27}\,\mathrm{g}$ is the Earth mass), which
%    is independent of the outer boundary
%    conditions and therefore independent of the location in the
%    solar nebula. This critical value for the core mass corresponds
%    closely to the cores of today's giant planets.

%    Although no hydrodynamical study has been available many workers
%    conjectured that a collapse or rapid contraction will ensue
%    after accumulating the critical mass. The main motivation for
%    this article
%    is to investigate the stability of the static envelope at the
%    critical mass. With this aim the local, linear stability of static
%    radiative gas  spheres is investigated on the basis of Baker's
%    (\citeyear{baker}) standard one-zone model. 

%    Phenomena similar to the ones described above for giant planet
%    formation have been found in hydrodynamical models concerning
%    star formation where protostellar cores explode
%    (Tscharnuter \citeyear{tscharnuter}, Balluch \citeyear{balluch}),
%    whereas earlier studies found quasi-steady collapse flows. The
%    similarities in the (micro)physics, i.e., constitutive relations of
%    protostellar cores and protogiant planets serve as a further
%    motivation for this study.

% By conducting this project, I aim to investigate these discrepancies and improve our understanding of the formation of dark matter halos. By comparing the halo mass function and density profiles derived from simulations with theoretical models, we can test and refine our understanding of cosmic structure formation and dark matter's role in it.

% This project will simulate different realizations of the universe, compute the dark matter halo mass function, and analyze the density profiles of halos to explore these theoretical and observational inconsistencies. Numerical simulations provide a powerful tool to investigate how well theoretical models match the real universe, guiding us toward more accurate representations of dark matter-driven structure formation.

\subsection{Introduction}\label{subsec: I introduction}

\subsection{Theory}\label{subsec: I theory}
% The scaled Hubble parameter is given by
% \begin{equation}
%   \mathcal{H} = aH 
%   = H_0\sqrt{\left(\Omega_{b0} + \Omega_\text{CDM0}\right)a^{-1} 
%   + \left(\Omega_{\gamma0} + \Omega_{\nu0}\right)a^{-2}
%   + \Omega_{k0} + \Omega_{\Lambda0}a^2},
% \end{equation}
% or in terms of our time variable $x=\log a$:
% \begin{equation}
%   \mathcal{H} = H_0
%   \sqrt{\left(\Omega_{b0} + \Omega_\text{CDM0}\right)e^{-x} 
%   + \left(\Omega_{\gamma0} + \Omega_{\nu0}\right)e^{-2x}
%   + \Omega_{k0} + \Omega_{\Lambda0}e^{2x}}.
% \end{equation}
% The first and second derivatives with respect to $x$ are then
% \begin{align}
%   \frac{d\mathcal{H}}{dx} &= \frac{H_0}{2}
%   \frac{-\left(\Omega_{b0} + \Omega_\text{CDM0}\right)e^{-x} - 2\left(\Omega_{\gamma0} + \Omega_{\nu0}\right)e^{-2x} + 2\Omega_{\Lambda0}e^{2x}}
%   {\sqrt{\left(\Omega_{b0} + \Omega_\text{CDM0}\right)e^{-x} 
%   + \left(\Omega_{\gamma0} + \Omega_{\nu0}\right)e^{-2x}
%   + \Omega_{k0} + \Omega_{\Lambda0}e^{2x}}},
%   \\
%   \frac{d^2\mathcal{H}}{dx^2} &= 
% \end{align}
The Hubble parameter is
\begin{equation}
  H = H_0\sqrt{\left(\Omega_{b0} + \Omega_\text{CDM0}\right)a^{-3} 
  + \left(\Omega_{\gamma0} + \Omega_{\nu0}\right)a^{-4}
  + \Omega_{k0}a^{-2} + \Omega_{\Lambda0}}. \label{eq:H}
\end{equation}
\colorbox{Plum}{TEXT}
The scaled Hubble parameter is given by
\begin{equation}
  \mathcal{H} = aH, \label{eq:Hp}
\end{equation}
Thus, if we define $\Omega_{m0} = \Omega_{b0}+\Omega_\text{CDM0}$ and $\Omega_{r0} = \Omega_{\gamma0}+\Omega_{\nu0}$, and use our chosen time variable $x=\log a$ instead of $a$, the expressions \eqref{eq:H} and \eqref{eq:Hp} can equivalently be written as
\begin{align}
  H &= H_0
  \sqrt{\Omega_{m0}e^{-3x} 
  + \Omega_{r0}e^{-4x}
  + \Omega_{k0}e^{-2x} + \Omega_{\Lambda0}},
  \\
  \mathcal{H} &= H_0
  \sqrt{\Omega_{m0}e^{-x} 
  + \Omega_{r0}e^{-2x}
  + \Omega_{k0} + \Omega_{\Lambda0}e^{2x}}.
\end{align}
The first and second derivatives of $\mathcal{H}$ with respect to $x$ are then \colorbox{Plum}{correct expressions?}
\begin{align*}
  \frac{d\mathcal{H}}{dx} &= \frac{H_0}{2}
  \frac{-\Omega_{m0}e^{-x} - 2\Omega_{r0}e^{-2x} + 2\Omega_{\Lambda0}e^{2x}}
  {\sqrt{\Omega_{m0}e^{-x} 
  + \Omega_{r0}e^{-2x}
  + \Omega_{k0} + \Omega_{\Lambda0}e^{2x}}},
  \\
  &= -\frac{H_0^2}{2\mathcal{H}}\left(\Omega_{m0}e^{-x} + 2\Omega_{r0}e^{-2x} - 2\Omega_{\Lambda0}e^{2x}\right), \numberthis
  \\
  \frac{d^2\mathcal{H}}{dx^2} &= \frac{H_0}{2}
  \left(\frac{\Omega_{m0}e^{-x} + 4\Omega_{r0}e^{-2x} + 4\Omega_{\Lambda0}e^{2x}}{\sqrt{\Omega_{m0}e^{-x} + \Omega_{r0}e^{-2x} + \Omega_{k0} + \Omega_{\Lambda0}e^{2x}}}\right.
  \\
  &\hspace{38pt}
  \left.+ \frac{1}{2}\frac{\Omega_{m0}e^{-x} + 2\Omega_{r0}e^{-2x} - 2\Omega_{\Lambda0}e^{2x}}{\left(\Omega_{m0}e^{-x} + \Omega_{r0}e^{-2x} + \Omega_{k0} + \Omega_{\Lambda0}e^{2x}\right)^{3/2}}\right),
  \\
  &= \frac{H_0^2}{2\mathcal{H}}\left(\Omega_{m0}e^{-x} + 4\Omega_{r0}e^{-2x} + 4\Omega_{\Lambda0}e^{2x}- \frac{1}{2\mathcal{H}}\frac{d\mathcal{H}}{dx}\right). \numberthis
\end{align*}


%--------------------------------------------------------------------
\subsection{Methods}\label{subsec: I methods}
% %-------------------------------------- Two column figure (place early!)
%    \begin{figure*}
%    \centering
%    %%%\includegraphics{empty.eps}
%    %%%\includegraphics{empty.eps}
%    \includegraphics[width=\textwidth]{density.pdf}
%    \caption{Adiabatic exponent $\Gamma_1$.
%                $\Gamma_1$ is plotted as a function of
%                $\lg$ internal energy $\mathrm{[erg\,g^{-1}]}$ and $\lg$
%                density $\mathrm{[g\,cm^{-3}]}$.}
%               \label{FigGam}%
%     \end{figure*}
% %
%    In this section the one-zone model of \citet{baker},
%    originally used to study the Cephe{\"{\i}}d pulsation mechanism, will
%    be briefly reviewed. The resulting stability criteria will be
%    rewritten in terms of local state variables, local timescales and
%    constitutive relations.

%    \citet{baker} investigates the stability of thin layers in
%    self-gravitating,
%    spherical gas clouds with the following properties:
%    \begin{itemize}
%       \item hydrostatic equilibrium,
%       \item thermal equilibrium,
%       \item energy transport by grey radiation diffusion.
%    \end{itemize}
%    For the one-zone-model Baker obtains necessary conditions
%    for dynamical, secular and vibrational (or pulsational)
%    stability (Eqs.\ (34a,\,b,\,c) in Baker \citeyear{baker}). Using Baker's
%    notation:
%    %%begin novalidate
%    \[
%       \begin{array}{lp{0.8\linewidth}}
%          M_{r}  & mass internal to the radius $r$     \\
%          m               & mass of the zone                    \\
%          r_0             & unperturbed zone radius             \\
%          \rho_0          & unperturbed density in the zone     \\
%          T_0             & unperturbed temperature in the zone \\
%          L_{r0}          & unperturbed luminosity              \\
%          E_{\mathrm{th}} & thermal energy of the zone
%       \end{array}
%    \]
%    %%end novalidate
% \noindent
%    and with the definitions of the \emph{local cooling time\/}
%    (see Fig.~\ref{FigGam})
%    \begin{equation}
%       \tau_{\mathrm{co}} = \frac{E_{\mathrm{th}}}{L_{r0}} \,,
%    \end{equation}
%    and the \emph{local free-fall time}
%    \begin{equation}
%       \tau_{\mathrm{ff}} =
%          \sqrt{ \frac{3 \pi}{32 G} \frac{4\pi r_0^3}{3 M_{\mathrm{r}}}
% }\,,
%    \end{equation}
%    Baker's $K$ and $\sigma_0$ have the following form:
%    \begin{eqnarray}
%       \sigma_0 & = & \frac{\pi}{\sqrt{8}}
%                      \frac{1}{ \tau_{\mathrm{ff}}} \\
%       K        & = & \frac{\sqrt{32}}{\pi} \frac{1}{\delta}
%                         \frac{ \tau_{\mathrm{ff}} }
%                              { \tau_{\mathrm{co}} }\,;
%    \end{eqnarray}
%    where $ E_{\mathrm{th}} \approx m (P_0/{\rho_0})$ has been used and
%    \begin{equation}
%    \begin{array}{l}
%       \delta = - \left(
%                     \frac{ \partial \ln \rho }{ \partial \ln T }
%                  \right)_P \\
%       e=mc^2
%    \end{array}
%    \end{equation}
%    is a thermodynamical quantity which is of order $1$ and equal to $1$
%    for nonreacting mixtures of classical perfect gases. The physical
%    meaning of $ \sigma_0 $ and $K$ is clearly visible in the equations
%    above. $\sigma_0$ represents a frequency of the order one per
%    free-fall time. $K$ is proportional to the ratio of the free-fall
%    time and the cooling time. Substituting into Baker's criteria, using
%    thermodynamic identities and definitions of thermodynamic quantities,
%    \begin{displaymath}
%       \Gamma_1      = \left( \frac{ \partial \ln P}{ \partial\ln \rho}
%                            \right)_{S}    \, , \;
%       \chi^{}_\rho  = \left( \frac{ \partial \ln P}{ \partial\ln \rho}
%                            \right)_{T}    \, , \;
%       \kappa^{}_{P} = \left( \frac{ \partial \ln \kappa}{ \partial\ln P}
%                            \right)_{T}
%    \end{displaymath}
%    \begin{displaymath}
%       \nabla_{\mathrm{ad}} = \left( \frac{ \partial \ln T}
%                              { \partial\ln P} \right)_{S} \, , \;
%       \chi^{}_T       = \left( \frac{ \partial \ln P}
%                              { \partial\ln T} \right)_{\rho} \, , \;
%       \kappa^{}_{T}   = \left( \frac{ \partial \ln \kappa}
%                              { \partial\ln T} \right)_{T}
%    \end{displaymath}
%    one obtains, after some pages of algebra, the conditions for
%    \emph{stability\/} given
%    below:
%    \begin{eqnarray}
%       \frac{\pi^2}{8} \frac{1}{\tau_{\mathrm{ff}}^2}
%                 ( 3 \Gamma_1 - 4 )
%          & > & 0 \label{ZSDynSta} \\
%       \frac{\pi^2}{\tau_{\mathrm{co}}
%                    \tau_{\mathrm{ff}}^2}
%                    \Gamma_1 \nabla_{\mathrm{ad}}
%                    \left[ \frac{ 1- 3/4 \chi^{}_\rho }{ \chi^{}_T }
%                           ( \kappa^{}_T - 4 )
%                         + \kappa^{}_P + 1
%                    \right]
%         & > & 0 \label{ZSSecSta} \\
%      \frac{\pi^2}{4} \frac{3}{\tau_{ \mathrm{co} }
%                               \tau_{ \mathrm{ff} }^2
%                              }
%          \Gamma_1^2 \, \nabla_{\mathrm{ad}} \left[
%                                    4 \nabla_{\mathrm{ad}}
%                                    - ( \nabla_{\mathrm{ad}} \kappa^{}_T
%                                      + \kappa^{}_P
%                                      )
%                                    - \frac{4}{3 \Gamma_1}
%                                 \right]
%         & > & 0   \label{ZSVibSta}
%    \end{eqnarray}
% %
%    For a physical discussion of the stability criteria see \citet{baker} or \citet{cox}.

%    We observe that these criteria for dynamical, secular and
%    vibrational stability, respectively, can be factorized into
%    \begin{enumerate}
%       \item a factor containing local timescales only,
%       \item a factor containing only constitutive relations and
%          their derivatives.
%    \end{enumerate}
%    The first factors, depending on only timescales, are positive
%    by definition. The signs of the left hand sides of the
%    inequalities~(\ref{ZSDynSta}), (\ref{ZSSecSta}) and (\ref{ZSVibSta})
%    therefore depend exclusively on the second factors containing
%    the constitutive relations. Since they depend only
%    on state variables, the stability criteria themselves are \emph{
%    functions of the thermodynamic state in the local zone}. The
%    one-zone stability can therefore be determined
%    from a simple equation of state, given for example, as a function
%    of density and
%    temperature. Once the microphysics, i.e.\ the thermodynamics
%    and opacities (see Table~\ref{KapSou}), are specified (in practice
%    by specifying a chemical composition) the one-zone stability can
%    be inferred if the thermodynamic state is specified.
%    The zone -- or in
%    other words the layer -- will be stable or unstable in
%    whatever object it is imbedded as long as it satisfies the
%    one-zone-model assumptions. Only the specific growth rates
%    (depending upon the time scales) will be different for layers
%    in different objects.

% %--------------------------------------------------- One column table
%    \begin{table}
%       \caption[]{Opacity sources.}
%          \label{KapSou}
%      $$ 
%          \begin{array}{p{0.5\linewidth}l}
%             \hline
%             \noalign{\smallskip}
%             Source      &  T / {[\mathrm{K}]} \\
%             \noalign{\smallskip}
%             \hline
%             \noalign{\smallskip}
%             Yorke 1979, Yorke 1980a & \leq 1700^{\mathrm{a}}     \\
% %           Yorke 1979, Yorke 1980a & \leq 1700             \\
%             Kr\"ugel 1971           & 1700 \leq T \leq 5000 \\
%             Cox \& Stewart 1969     & 5000 \leq             \\
%             \noalign{\smallskip}
%             \hline
%          \end{array}
%      $$ 
%    \end{table}
% %
%    We will now write down the sign (and therefore stability)
%    determining parts of the left-hand sides of the inequalities
%    (\ref{ZSDynSta}), (\ref{ZSSecSta}) and (\ref{ZSVibSta}) and thereby
%    obtain \emph{stability equations of state}.

%    The sign determining part of inequality~(\ref{ZSDynSta}) is
%    $3\Gamma_1 - 4$ and it reduces to the
%    criterion for dynamical stability
%    \begin{equation}
%      \Gamma_1 > \frac{4}{3}\,\cdot
%    \end{equation}
%    Stability of the thermodynamical equilibrium demands
%    \begin{equation}
%       \chi^{}_\rho > 0, \;\;  c_v > 0\, ,
%    \end{equation}
%    and
%    \begin{equation}
%       \chi^{}_T > 0
%    \end{equation}
%    holds for a wide range of physical situations.
%    With
%    \begin{eqnarray}
%       \Gamma_3 - 1 = \frac{P}{\rho T} \frac{\chi^{}_T}{c_v}&>&0\\
%       \Gamma_1     = \chi_\rho^{} + \chi_T^{} (\Gamma_3 -1)&>&0\\
%       \nabla_{\mathrm{ad}}  = \frac{\Gamma_3 - 1}{\Gamma_1}         &>&0
%    \end{eqnarray}
%    we find the sign determining terms in inequalities~(\ref{ZSSecSta})
%    and (\ref{ZSVibSta}) respectively and obtain the following form
%    of the criteria for dynamical, secular and vibrational
%    \emph{stability}, respectively:
%    \begin{eqnarray}
%       3 \Gamma_1 - 4 =: S_{\mathrm{dyn}}      > & 0 & \label{DynSta}  \\
% %
%       \frac{ 1- 3/4 \chi^{}_\rho }{ \chi^{}_T } ( \kappa^{}_T - 4 )
%          + \kappa^{}_P + 1 =: S_{\mathrm{sec}} > & 0 & \label{SecSta} \\
% %
%       4 \nabla_{\mathrm{ad}} - (\nabla_{\mathrm{ad}} \kappa^{}_T
%                              + \kappa^{}_P)
%                              - \frac{4}{3 \Gamma_1} =: S_{\mathrm{vib}}
%                                       > & 0\,.& \label{VibSta}
%    \end{eqnarray}
%    The constitutive relations are to be evaluated for the
%    unperturbed thermodynamic state (say $(\rho_0, T_0)$) of the zone.
%    We see that the one-zone stability of the layer depends only on
%    the constitutive relations $\Gamma_1$,
%    $\nabla_{\mathrm{ad}}$, $\chi_T^{},\,\chi_\rho^{}$,
%    $\kappa_P^{},\,\kappa_T^{}$.
%    These depend only on the unperturbed
%    thermodynamical state of the layer. Therefore the above relations
%    define the one-zone-stability equations of state
%    $S_{\mathrm{dyn}},\,S_{\mathrm{sec}}$
%    and $S_{\mathrm{vib}}$. See Fig.~\ref{FigVibStab} for a picture of
%    $S_{\mathrm{vib}}$. Regions of secular instability are
%    listed in Table~1.

% %
% %                                                One column figure
% %----------------------------------------------------------------- 
%    \begin{figure}
%    \centering
%    %%%\includegraphics[width=3cm]{empty.eps}
%       \caption{Vibrational stability equation of state
%                $S_{\mathrm{vib}}(\lg e, \lg \rho)$.
%                $>0$ means vibrational stability.
%               }
%          \label{FigVibStab}
%    \end{figure}
% %-----------------------------------------------------------------
% \subsection{Background theory}
% \subsubsection{Halo mass function}
% \subsubsection{Press-Schechter formalism}
% \subsubsection{Navarro-Frenk-White profile}

% Cosmological N-body simulations are essential because they allow us to explore non-linear gravitational interactions and structure formation in ways that analytical methods cannot fully capture. Despite the success of analytical models like Press-Schechter, discrepancies between these theoretical predictions and observations often arise. For example, the overabundance of small halos predicted by the Press-Schechter formalism compared to what is observed in the real universe, also known as the ``missing satellite problem'', and the differences in the inner structure of halos, where simulations show more complex dynamics that deviate from simple models. \colorbox{Plum}{observations? shorten? move to intro?}


\subsection{Results \& Discussions}\label{subsec: I results}
% \begin{figure*}
%   \centering
%   \includegraphics[width=\textwidth]{hmf_candidates.pdf}
%   \caption{HMF values calculated at $z=0$ for candidates 1, 2, 4, 7, 11 and 8 (the ``universe'' analyzed in this work), as well as for Sijing Shen. The Press-Schechter (solid line) and Sheth-Tormen (dashed line) fits are also plotted, and are the same for all realizations. In Shen's simulation, a box of size of $L=40\:\text{Mpc}$ containing $128^3$ particles was used instead.}\label{fig:hmf cand}
% \end{figure*}

\subsection{Conclusions}\label{subsec: I conclusions}

% \begin{acknowledgements}
%       Part of this work was supported by the German
%       \emph{Deut\-sche For\-schungs\-ge\-mein\-schaft, DFG\/} project
%       number Ts~17/2--1.
% \end{acknowledgements}

% WARNING
%-------------------------------------------------------------------
% Please note that we have included the references to the file aa.dem in
% order to compile it, but we ask you to:
%
% - use BibTeX with the regular commands:
%   \bibliographystyle{aa} % style aa.bst
%   \bibliography{Yourfile} % your references Yourfile.bib
%
% - join the .bib files when you upload your source files
%-------------------------------------------------------------------

  \cite{Armadillo}
  \bibliographystyle{aa} % style aa.bst
  \bibliography{references} % your references Yourfile.bib


\end{document}















% %
% %%%%%%%%%%%%%%%%%%%%%%%%%%%%%%%%%%%%%%%%%%%%%%%%%%%%%%%%%%%%%%
% Example below of non-structurated natbib references  
% To use the v8.3 macros with this form of composition of bibliography, 
% the option "bibyear" should be added to the command line 
% "\documentclass[bibyear]{aa}".
% %%%%%%%%%%%%%%%%%%%%%%%%%%%%%%%%%%%%%%%%%%%%%%%%%%%%%%%%%%%%%%

% \begin{thebibliography}{}

%   \bibitem[1966]{baker} Baker, N. 1966,
%       in Stellar Evolution,
%       ed.\ R. F. Stein,\& A. G. W. Cameron
%       (Plenum, New York) 333

%    \bibitem[1988]{balluch} Balluch, M. 1988,
%       A\&A, 200, 58

%    \bibitem[1980]{cox} Cox, J. P. 1980,
%       Theory of Stellar Pulsation
%       (Princeton University Press, Princeton) 165

%    \bibitem[1969]{cox69} Cox, A. N.,\& Stewart, J. N. 1969,
%       Academia Nauk, Scientific Information 15, 1

%    \bibitem[1980]{mizuno} Mizuno H. 1980,
%       Prog. Theor. Phys., 64, 544
   
%    \bibitem[1987]{tscharnuter} Tscharnuter W. M. 1987,
%       A\&A, 188, 55
  
%    \bibitem[1992]{terlevich} Terlevich, R. 1992, in ASP Conf. Ser. 31, 
%       Relationships between Active Galactic Nuclei and Starburst Galaxies, 
%       ed. A. V. Filippenko, 13

%    \bibitem[1980a]{yorke80a} Yorke, H. W. 1980a,
%       A\&A, 86, 286

%    \bibitem[1997]{zheng} Zheng, W., Davidsen, A. F., Tytler, D. \& Kriss, G. A.
%       1997, preprint
% \end{thebibliography}
% %
% %%%%%%%%%%%%%%%%%%%%%%%%%%%%%%%%%%%%%%%%%%%%%%%%%%%%%%%%%%%%%%
% Examples for figures using graphicx
% A guide "Using Imported Graphics in LaTeX2e"  (Keith Reckdahl)
% is available on a lot of LaTeX public servers or ctan mirrors.
% The file is : epslatex.pdf 
% %%%%%%%%%%%%%%%%%%%%%%%%%%%%%%%%%%%%%%%%%%%%%%%%%%%%%%%%%%%%%%

% %-------------------------------------------------------------
% %                 A figure as large as the width of the column
% %-------------------------------------------------------------
%    \begin{figure}
%    \centering
%    \includegraphics[width=\hsize]{empty.eps}
%       \caption{Vibrational stability equation of state
%                $S_{\mathrm{vib}}(\lg e, \lg \rho)$.
%                $>0$ means vibrational stability.
%               }
%          \label{FigVibStab}
%    \end{figure}
% %
% %-------------------------------------------------------------
% %                                    One column rotated figure
% %-------------------------------------------------------------
%    \begin{figure}
%    \centering
%    \includegraphics[angle=-90,width=3cm]{empty.eps}
%       \caption{Vibrational stability equation of state
%                $S_{\mathrm{vib}}(\lg e, \lg \rho)$.
%                $>0$ means vibrational stability.
%               }
%          \label{FigVibStab}
%    \end{figure}
% %
% %-------------------------------------------------------------
% %                        Figure with caption on the right side 
% %-------------------------------------------------------------
%    \begin{figure}
%    \sidecaption
%    \includegraphics[width=3cm]{empty.eps}
%       \caption{Vibrational stability equation of state
%                $S_{\mathrm{vib}}(\lg e, \lg \rho)$.
%                $>0$ means vibrational stability.
%               }
%          \label{FigVibStab}
%    \end{figure}
% %
% %-------------------------------------------------------------
% %                                Figure with a new BoundingBox 
% %-------------------------------------------------------------
%    \begin{figure}
%    \centering
%    \includegraphics[bb=10 20 100 300,width=3cm,clip]{empty.eps}
%       \caption{Vibrational stability equation of state
%                $S_{\mathrm{vib}}(\lg e, \lg \rho)$.
%                $>0$ means vibrational stability.
%               }
%          \label{FigVibStab}
%    \end{figure}
% %
% %-------------------------------------------------------------
% %                                      The "resizebox" command 
% %-------------------------------------------------------------
%    \begin{figure}
%    \resizebox{\hsize}{!}
%             {\includegraphics[bb=10 20 100 300,clip]{empty.eps}
%       \caption{Vibrational stability equation of state
%                $S_{\mathrm{vib}}(\lg e, \lg \rho)$.
%                $>0$ means vibrational stability.
%               }
%          \label{FigVibStab}
%    \end{figure}
% %
% %-------------------------------------------------------------
% %                                             Two column Figure 
% %-------------------------------------------------------------
%    \begin{figure*}
%    \resizebox{\hsize}{!}
%             {\includegraphics[bb=10 20 100 300,clip]{empty.eps}
%       \caption{Vibrational stability equation of state
%                $S_{\mathrm{vib}}(\lg e, \lg \rho)$.
%                $>0$ means vibrational stability.
%               }
%          \label{FigVibStab}
%    \end{figure*}
% %
% %-------------------------------------------------------------
% %                                             Simple A&A Table
% %-------------------------------------------------------------
% %
% \begin{table}
% \caption{Nonlinear Model Results}             % title of Table
% \label{table:1}      % is used to refer this table in the text
% \centering                          % used for centering table
% \begin{tabular}{c c c c}        % centered columns (4 columns)
% \hline\hline                 % inserts double horizontal lines
% HJD & $E$ & Method\#2 & Method\#3 \\    % table heading 
% \hline                        % inserts single horizontal line
%    1 & 50 & $-837$ & 970 \\      % inserting body of the table
%    2 & 47 & 877    & 230 \\
%    3 & 31 & 25     & 415 \\
%    4 & 35 & 144    & 2356 \\
%    5 & 45 & 300    & 556 \\ 
% \hline                                   %inserts single line
% \end{tabular}
% \end{table}
% %
% %-------------------------------------------------------------
% %                                             Two column Table 
% %-------------------------------------------------------------
% %
% \begin{table*}
% \caption{Nonlinear Model Results}             
% \label{table:1}      
% \centering          
% \begin{tabular}{c c c c l l l }     % 7 columns 
% \hline\hline       
%                       % To combine 4 columns into a single one 
% HJD & $E$ & Method\#2 & \multicolumn{4}{c}{Method\#3}\\ 
% \hline                    
%    1 & 50 & $-837$ & 970 & 65 & 67 & 78\\  
%    2 & 47 & 877    & 230 & 567& 55 & 78\\
%    3 & 31 & 25     & 415 & 567& 55 & 78\\
%    4 & 35 & 144    & 2356& 567& 55 & 78 \\
%    5 & 45 & 300    & 556 & 567& 55 & 78\\
% \hline                  
% \end{tabular}
% \end{table*}
% %
% %-------------------------------------------------------------
% %                                          Table with notes 
% %-------------------------------------------------------------
% %
% % A single note
% \begin{table}
% \caption{\label{t7}Spectral types and photometry for stars in the
%   region.}
% \centering
% \begin{tabular}{lccc}
% \hline\hline
% Star&Spectral type&RA(J2000)&Dec(J2000)\\
% \hline
% 69           &B1\,V     &09 15 54.046 & $-$50 00 26.67\\
% 49           &B0.7\,V   &*09 15 54.570& $-$50 00 03.90\\
% LS~1267~(86) &O8\,V     &09 15 52.787&11.07\\
% 24.6         &7.58      &1.37 &0.20\\
% \hline
% LS~1262      &B0\,V     &09 15 05.17&11.17\\
% MO 2-119     &B0.5\,V   &09 15 33.7 &11.74\\
% LS~1269      &O8.5\,V   &09 15 56.60&10.85\\
% \hline
% \end{tabular}
% \tablefoot{The top panel shows likely members of Pismis~11. The second
% panel contains likely members of Alicante~5. The bottom panel
% displays stars outside the clusters.}
% \end{table}
% %
% % More notes
% %
% \begin{table}
% \caption{\label{t7}Spectral types and photometry for stars in the
%   region.}
% \centering
% \begin{tabular}{lccc}
% \hline\hline
% Star&Spectral type&RA(J2000)&Dec(J2000)\\
% \hline
% 69           &B1\,V     &09 15 54.046 & $-$50 00 26.67\\
% 49           &B0.7\,V   &*09 15 54.570& $-$50 00 03.90\\
% LS~1267~(86) &O8\,V     &09 15 52.787&11.07\tablefootmark{a}\\
% 24.6         &7.58\tablefootmark{1}&1.37\tablefootmark{a}   &0.20\tablefootmark{a}\\
% \hline
% LS~1262      &B0\,V     &09 15 05.17&11.17\tablefootmark{b}\\
% MO 2-119     &B0.5\,V   &09 15 33.7 &11.74\tablefootmark{c}\\
% LS~1269      &O8.5\,V   &09 15 56.60&10.85\tablefootmark{d}\\
% \hline
% \end{tabular}
% \tablefoot{The top panel shows likely members of Pismis~11. The second
% panel contains likely members of Alicante~5. The bottom panel
% displays stars outside the clusters.\\
% \tablefoottext{a}{Photometry for MF13, LS~1267 and HD~80077 from
% Dupont et al.}
% \tablefoottext{b}{Photometry for LS~1262, LS~1269 from
% Durand et al.}
% \tablefoottext{c}{Photometry for MO2-119 from
% Mathieu et al.}
% }
% \end{table}
% %
% %-------------------------------------------------------------
% %                                       Table with references 
% %-------------------------------------------------------------
% %
% \begin{table*}[h]
%  \caption[]{\label{nearbylistaa2}List of nearby SNe used in this work.}
% \begin{tabular}{lccc}
%  \hline \hline
%   SN name &
%   Epoch &
%  Bands &
%   References \\
%  &
%   (with respect to $B$ maximum) &
%  &
%  \\ \hline
% 1981B   & 0 & {\it UBV} & 1\\
% 1986G   &  $-$3, $-$1, 0, 1, 2 & {\it BV}  & 2\\
% 1989B   & $-$5, $-$1, 0, 3, 5 & {\it UBVRI}  & 3, 4\\
% 1990N   & 2, 7 & {\it UBVRI}  & 5\\
% 1991M   & 3 & {\it VRI}  & 6\\
% \hline
% \noalign{\smallskip}
% \multicolumn{4}{c}{ SNe 91bg-like} \\
% \noalign{\smallskip}
% \hline
% 1991bg   & 1, 2 & {\it BVRI}  & 7\\
% 1999by   & $-$5, $-$4, $-$3, 3, 4, 5 & {\it UBVRI}  & 8\\
% \hline
% \noalign{\smallskip}
% \multicolumn{4}{c}{ SNe 91T-like} \\
% \noalign{\smallskip}
% \hline
% 1991T   & $-$3, 0 & {\it UBVRI}  &  9, 10\\
% 2000cx  & $-$3, $-$2, 0, 1, 5 & {\it UBVRI}  & 11\\ %
% \hline
% \end{tabular}
% \tablebib{(1)~\citet{branch83};
% (2) \citet{phillips87}; (3) \citet{barbon90}; (4) \citet{wells94};
% (5) \citet{mazzali93}; (6) \citet{gomez98}; (7) \citet{kirshner93};
% (8) \citet{patat96}; (9) \citet{salvo01}; (10) \citet{branch03};
% (11) \citet{jha99}.
% }
% \end{table}
% %-------------------------------------------------------------
% %                      A rotated Two column Table in landscape  
% %-------------------------------------------------------------
% \begin{sidewaystable*}
% \caption{Summary for ISOCAM sources with mid-IR excess 
% (YSO candidates).}\label{YSOtable}
% \centering
% \begin{tabular}{crrlcl} 
% \hline\hline             
% ISO-L1551 & $F_{6.7}$~[mJy] & $\alpha_{6.7-14.3}$ 
% & YSO type$^{d}$ & Status & Comments\\
% \hline
%   \multicolumn{6}{c}{\it New YSO candidates}\\ % To combine 6 columns into a single one
% \hline
%   1 & 1.56 $\pm$ 0.47 & --    & Class II$^{c}$ & New & Mid\\
%   2 & 0.79:           & 0.97: & Class II ?     & New & \\
%   3 & 4.95 $\pm$ 0.68 & 3.18  & Class II / III & New & \\
%   5 & 1.44 $\pm$ 0.33 & 1.88  & Class II       & New & \\
% \hline
%   \multicolumn{6}{c}{\it Previously known YSOs} \\
% \hline
%   61 & 0.89 $\pm$ 0.58 & 1.77 & Class I & \object{HH 30} & Circumstellar disk\\
%   96 & 38.34 $\pm$ 0.71 & 37.5& Class II& MHO 5          & Spectral type\\
% \hline
% \end{tabular}
% \end{sidewaystable*}
% %-------------------------------------------------------------
% %                      A rotated One column Table in landscape  
% %-------------------------------------------------------------
% \begin{sidewaystable}
% \caption{Summary for ISOCAM sources with mid-IR excess 
% (YSO candidates).}\label{YSOtable}
% \centering
% \begin{tabular}{crrlcl} 
% \hline\hline             
% ISO-L1551 & $F_{6.7}$~[mJy] & $\alpha_{6.7-14.3}$ 
% & YSO type$^{d}$ & Status & Comments\\
% \hline
%   \multicolumn{6}{c}{\it New YSO candidates}\\ % To combine 6 columns into a single one
% \hline
%   1 & 1.56 $\pm$ 0.47 & --    & Class II$^{c}$ & New & Mid\\
%   2 & 0.79:           & 0.97: & Class II ?     & New & \\
%   3 & 4.95 $\pm$ 0.68 & 3.18  & Class II / III & New & \\
%   5 & 1.44 $\pm$ 0.33 & 1.88  & Class II       & New & \\
% \hline
%   \multicolumn{6}{c}{\it Previously known YSOs} \\
% \hline
%   61 & 0.89 $\pm$ 0.58 & 1.77 & Class I & \object{HH 30} & Circumstellar disk\\
%   96 & 38.34 $\pm$ 0.71 & 37.5& Class II& MHO 5          & Spectral type\\
% \hline
% \end{tabular}
% \end{sidewaystable}
% %
% %-------------------------------------------------------------
% %                              Table longer than a single page  
% %-------------------------------------------------------------
% % All long tables will be placed automatically at the end of the document
% %
% \longtab{
% \begin{longtable}{lllrrr}
% \caption{\label{kstars} Sample stars with absolute magnitude}\\
% \hline\hline
% Catalogue& $M_{V}$ & Spectral & Distance & Mode & Count Rate \\
% \hline
% \endfirsthead
% \caption{continued.}\\
% \hline\hline
% Catalogue& $M_{V}$ & Spectral & Distance & Mode & Count Rate \\
% \hline
% \endhead
% \hline
% \endfoot
% %%
% Gl 33    & 6.37 & K2 V & 7.46 & S & 0.043170\\
% Gl 66AB  & 6.26 & K2 V & 8.15 & S & 0.260478\\
% Gl 68    & 5.87 & K1 V & 7.47 & P & 0.026610\\
%          &      &      &      & H & 0.008686\\
% Gl 86 
% \footnote{Source not included in the HRI catalog. See Sect.~5.4.2 for details.}
%          & 5.92 & K0 V & 10.91& S & 0.058230\\
% \end{longtable}
% }
% %
% %-------------------------------------------------------------
% %                              Table longer than a single page
% %                                            and in landscape, 
% %                    in the preamble, use: \usepackage{lscape}
% %-------------------------------------------------------------

% % All long tables will be placed automatically at the end of the document
% %
% \longtab{
% \begin{landscape}
% \begin{longtable}{lllrrr}
% \caption{\label{kstars} Sample stars with absolute magnitude}\\
% \hline\hline
% Catalogue& $M_{V}$ & Spectral & Distance & Mode & Count Rate \\
% \hline
% \endfirsthead
% \caption{continued.}\\
% \hline\hline
% Catalogue& $M_{V}$ & Spectral & Distance & Mode & Count Rate \\
% \hline
% \endhead
% \hline
% \endfoot
% %%
% Gl 33    & 6.37 & K2 V & 7.46 & S & 0.043170\\
% Gl 66AB  & 6.26 & K2 V & 8.15 & S & 0.260478\\
% Gl 68    & 5.87 & K1 V & 7.47 & P & 0.026610\\
%          &      &      &      & H & 0.008686\\
% Gl 86
% \footnote{Source not included in the HRI catalog. See Sect.~5.4.2 for details.}
%          & 5.92 & K0 V & 10.91& S & 0.058230\\
% \end{longtable}
% \end{landscape}
% }
% %
% %-------------------------------------------------------------
% %               Appendices have to be placed at the end, after
% %                                        \end{thebibliography}
% %-------------------------------------------------------------
% \end{thebibliography}

% \begin{appendix} %First appendix
% \section{Background galaxy number counts and shear noise-levels}
% Because the optical images used in this analysis...
% \begin{figure*}%f1
% \includegraphics[width=10.9cm]{1787f23.eps}
% \caption{Shown in greyscale is a...}
% \label{cl12301}
% \end{figure*}

% In this case....
% \begin{figure*}
% \centering
% \includegraphics[width=16.4cm,clip]{1787f24.ps}
% \caption{Plotted above...}
% \label{appfig}
% \end{figure*}

% Because the optical images...

% \section{Title of Second appendix.....} %Second appendix
% These studies, however, have faced...
% \begin{table}
% \caption{Complexes characterisation.}\label{starbursts}
% \centering
% \begin{tabular}{lccc}
% \hline \hline
% Complex & $F_{60}$ & 8.6 &  No. of  \\
% ...
% \hline
% \end{tabular}
% \end{table}

% The second method produces...
% \end{appendix}
% %
% %
% \end{document}

% %
% %-------------------------------------------------------------
% %          For the appendices, table longer than a single page
% %-------------------------------------------------------------

% % Table will be print automatically at the end of the document, 
% % after the whole appendices

% \begin{appendix} %First appendix
% \section{Background galaxy number counts and shear noise-levels}

% % In the appendices do not forget to put the counter of the table 
% % as an option

% \longtab[1]{
% \begin{longtable}{lrcrrrrrrrrl}
% \caption{Line data and abundances ...}\\
% \hline
% \hline
% Def & mol & Ion & $\lambda$ & $\chi$ & $\log gf$ & N & e &  rad & $\delta$ & $\delta$ 
% red & References \\
% \hline
% \endfirsthead
% \caption{Continued.} \\
% \hline
% Def & mol & Ion & $\lambda$ & $\chi$ & $\log gf$ & B & C &  rad & $\delta$ & $\delta$ 
% red & References \\
% \hline
% \endhead
% \hline
% \endfoot
% \hline
% \endlastfoot
% A & CH & 1 &3638 & 0.002 & $-$2.551 &  &  &  & $-$150 & 150 &  Jorgensen et al. (1996) \\                    
% \end{longtable}
% }% End longtab
% \end{appendix}

% %-------------------------------------------------------------
% %                   For appendices and landscape, large table:
% %                    in the preamble, use: \usepackage{lscape}
% %-------------------------------------------------------------

% \begin{appendix} %First appendix
% %
% \longtab[1]{
% \begin{landscape}
% \begin{longtable}{lrcrrrrrrrrl}
% ...
% \end{longtable}
% \end{landscape}
% }% End longtab
% \end{appendix}

% %%%% End of aa.dem